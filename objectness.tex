\section{Pre-train features by learning ``objectness''}
\label{sec:objectness}

%%%%%%%% Table Learn Pixel Objectness for pre-training
\begin{table}[t]
\resizebox{\columnwidth}{!}{
\centering
\begin{tabular}{l|llll}
Initial Repr.  & pixel acc. & mean acc. & mean IU & f.w. IU \\
\hline
ImageNetModel         &  &  &  & \\
CompleteCategory      &  &  &  & \\
PixelObjectness       &  &  & & \\
RandomCategory        &  &  &  & \\
FromScratch           &  &  & & \\
\end{tabular}
}
\caption{Performances of FCN with Alexnet trained to segment 5 categories from the PASCAL VOC2011 dataset with different representation initializations.
% \textit{ImageNetModel} represents the ImageNet pre-trained model;
% \textit{FromScratch} indicates that the representation is randomly initialization using Xavier Initialization;
\textit{CompleteCategory} is the model pre-trained to segment the other 15 categories from the PASCAL VOC2011 dataset; The \textit{PixelObjectness} model was pre-trained to distinguish the instance against the background; The \textit{RandomCategory} model was pre-trained with instances assigned random categories from the other 15 categories.}
\label{tab:objectness}
\end{table}


%%%%%%%% FIGURE Varying positive annotating percetage
\begin{figure}[t]
\centering
\fbox{\rule{0pt}{2in} \rule{0.9\linewidth}{0pt}}
  %  \includegraphics[width=0.95\linewidth]{img/}
\caption{Varying the number of categories while pre-training the representation and the pre-trained weights were fine-tuned to segment 5 categories from the PASCAL VOC2011 dataset.}
\label{fig:categories}
\end{figure}
