\section{Feature transferability with label noises}
\label{sec:robustness}

%%%%%%%%%%%%%%%%%%%%%%%%%%%%%%%%%%%%%%%%%%%%%%%%%%%%%%%%%%%%%%%%%%%%%%
%%%%%%%% TEXT Formulation of transferability, synthesized noises
%%%%%%%%%%%%%%%%%%%%%%%%%%%%%%%%%%%%%%%%%%%%%%%%%%%%%%%%%%%%%%%%%%%%%%

\subsection{Problem Formulation}
\label{subsec:formulation}

\paragraph{Semantic Segmentation}

A deep learning model for semantic segmentation normally consists of two main functions: a CNN feature extractor $G$ that extracts hierarchical feature maps $h$ from images $x$, followed by a classifier $H$ that generates pixel-by-pixel prediction to fit labels $y$.
Together they form a segmentation model $F$ to predict class probabilities for each of the pixels in a given image $x$:
$$P(y \vert x; F) = F(x) = H(G(x))$$
% $R^{h \times w \times c} \rightarrow R^{d}$
% $H: R^{d} \rightarrow R^{h \times w \times k}$
% $R^{h \times w \times c} \rightarrow R^{h \times w \times k}$,
% where $h, w$ are image height and weight respectively; $c$ is the number of image channels and $k$ is the number of classes; $d$ is the total number of extracted features.
% Prediction is then given by:
% $$P(y \vert x) = \sigma(F(x))$$
% where $\sigma$ is the sofmtax function.

\paragraph{Feature Transferability}
Feature extractor $G$ can be transferred from a pre-trained model.
The transferred feature extractor $\tilde{G}$ pre-trained with $\tilde{y}$ should be $G^{\ast}$ pre-trained with $y^{\ast}$ and fine-tune with the dataset of interest.
The pre-trained model is normally an ImageNet model trained by the object recognition task.
Alternatively, one can also obtain a pre-trained model with semantic segmentation task if there exists a segmentation dataset in a more similar domain than ImageNet.
The transferability of a pre-trained feature extractor can be measured by performance achieved by $F$ with transferred $G'$ on the test set.

Supposing a pre-training segmentation dataset have both noisy labels $\tilde{y}$ and true labels $y$ available, the influence of label noises can be studied by comparing the transferability of feature extractors learned with $y$ and with $\tilde{y}$. It is difficult to find such a dataset with both clean and noisy labels, so we tried to synthesize segmentation noises with well-annotated labels.

\subsection{Label noise synthesization}
\label{subsec:noises}

\noindent \textit{How noises synthesized}
\noindent
For segmentation problems, each pixel (or voxel for 3D segmentation) of an training image has a label assigned to one of the pre-defined categories.
Supposing there are $K$ pre-defined categories, the label of pixel ${ij}$
\[
  y_{ij} =
    \begin{cases}
      1 < k < K, & \text{for foreground pixels} \\
      0, & \text{for background pixels}
    \end{cases}
\]
where $1 < i < h, 1 < j < w$ and $i,j,k \in \mathrm{Z}^+$.
% Pixels with $k=0$ are also  \textit{unannotated} since they were not assigned to one of the predefined categories.

A straightforward way to synthesize noisy labels is to corrupt stochastically true labels for each pixel with a corruption model.
The corruption model describes the probability of an observed label conditioning on the the true label $y_{ij}$, the image $x$ and the label errorness $e_{ij}$:
$$p(\tilde{y_{ij}} \vert x, y_{ij}, e_{ij})$$
where the binary error occurence $e_{ij}$ depends on the inputs $x$ and true labels $y$.
Given a corruption model and true labels, one can stochastically synthesized the corresponding noisy labels.

\noindent \textit{Clarity for noises considered.}
% In the above model, label errorness for pixels is assumed to be independent to each other despite the fact that neighboring pixels are likely to have the same label in practice.
%Conditional random fields (CRF) were introduced to interpret the neighboring pixel dependence.\cite{chen2016deeplab}
In our works, we considered three types of noises for a semantic segmentation problem, \textit{mis-segmentation}, \textit{misclassification} and \textit{inexhaustive segmentation}, and modified the above corruption model for each type of noise accordingly.
Note that all these label errors apply to the whole segment instead of to individual pixels.
That is pixels for the same segments will have the same true labels and observed labels.
Therefore, the above corruption model hides the spatial dependence of $e_{ij}$ from expressions.
% It would be of more value to use a real dataset for such errors than to synthesize.
% The occurrence of error of each pixel is no longer simply dependent on which segments it belongs to but instead dependent on the neighboring pixels.
% How to model the pixel label errorness conditioning on neightboring pixels is not the focus of this paper and is left for future studies.

% $p(\tilde{y_{ij}},x) = p(\tilde{y_{ij}} \vert x, y, e_{ij}) p(e_{ij} \vert x,y)$.


\paragraph{False segmentations}
% Mis-segmentation denotes the wrongly segmented objects for categories that are semantically meaningful but are not predefined.
% For example, a toy dog can be misannotated as a dog, assuming that ``dog'' is predefined and ``toy dog'' is not.

In the presence of false segmentations, pixel labels of segment $S$ transit from $0$ to $k$ with probability
$$p(\tilde{y_{ij}}=k \vert x, y_{ij}=0), ij \in S $$
% $\{\tilde{y_{kl}}, kl \in P \setminus \{ij\} \}$
% where P is all the pixels in image $x$, and $ij$, $kl$ are both pixels in the image.
The dependence of observed pixel labels $\tilde{y_{ij}}$ on the original image $x$ interpret the premise that mis-segmentation would only happen to semantically meaningful segments in an image.
It is natural to include this premise because semantically meaningless partitions of an image are less likely to be segmented by an annotator.
However, it is difficult to estimate the above proability in practice because it is conditioning on the semantic meaning of $x$.
Therefore, we synthesized mis-segmentation errors by selecting part of the categories as non-target categories so that instances of these categories should have zero labels for correct segmentations.
We can then misannotate these non-target instances stochastically with a simplified probability $p(\tilde{y_{ij}}=k \vert y_{ij}=0)=p_k$ without  interpreting semantic meaning of $x$ in probability.
Note that $p_k$ sums up to 1 for all classes $\sum_0^K p_k = 1$.

\paragraph{Misclassification}
% Different from mis-segmentation, misclassification error means labels were misclassified between pre-defined categories.
% For example, cats may be misclassified as dogs occasionally if both ``cat'' and ``dog'' are target classes.

In the presence of misclassification, pixel labels of segment $S$ are transited from $k$ to $j$ stochastically with probability:
$$p(\tilde{y_{ij}}=j \vert y_{ij}=k) = p_{jk}, ij \in S, k,j \in [1,K]$$
where $\sum_{j=1}^{K}p_{jk}=1$.
We assumed misclassification error is independent of the exact shape and appearance of the objects, i.e.information from $x$.
This model is often called \textit{noisy at random} \cite{frenay2014classification}.
This assumption does not hold in every cases of practice, for example, some instances can be more likely to be misclassifified due to its ambiguity in shapes or apperances.
But the difficulty of modeling the depence of $x$ leads to simply assuming an input indepence.
Given the class transition probabilities, one can easily synthesize noisy annotations including misclassification errors given a well-annotated segmentation dataset.

\paragraph{Inexhaustive segmentation}

Pixels of an unsegmented object $S$ have labels flipped from $k$ to $0$ with probability:
$$p(\tilde{y_{ij}}=0\vert y_{ij}=k) = q_k, ij \in S, k \in [1,K]$$
In words, an instance of category $k$ is left unsegmented stochastically with probability $q_k$.
The probability of correctly segmented in annotations is then:
$$p(\tilde{y_{ij}}=k\vert y_{ij}=k) = 1-q_k, ij \in S, k \in [1,K]$$
% Again we assumed inexhaustive segmentations to be noisy at random (NAR).


% Given that we believe any annotated instance provide information, all the foreground pixels that correspond to the annotated instances become reliable and the background pixels may contain both the true background pixels and object pixels unannotated.
% That satisfies a Positive and Unlabeled learning setup where the training dataset contains only the positive examples and unlabeled examples that are the mixed of the positive samples and negative samples.

% But dogs and cats have visual features: they are both furry, have similiar shape and appearance, etc.
% These similiarities in features between dogs and cats may correspond to some interm features when a toy dog was misannotated as a dog.
% If we consider further lower features, for exmaple those for detecting edges and corners, they are commonly transferable to many categories, not necessarily ones that look alike.
% dissimilar categories (man-made classes and natural classes)
%If we constraint ourselves to perceive only part of the dog and the toy dog, for example the ears, it may become difficult even for us to distinguish the two because given only the local information there is not much difference between them.
% The generality of the low-level features can introduce the robustness to mis-segmentations regarding the transferability when we want to transfer hieraychical features for semantic segmentation in the presence misannotated objects because the low-level features are not dependent on the exact category assigned to the objects.
% This annotation error robustness for the low-level features could result in noise robustness for the multiple level pre-trained weights for semantic segmentation with noisy labels.
% In Section \ref{sec:objectness}, we tested whether object mis-segmentation had an impact on the transfehen we transfer the learned features to a new dataset with new categories.
% That leads us to the following research question:
% \begin{enumerate}
%   \item How do mis-segmentation and misclassification of instances influence the ``transferability'' of the learned features respectively?
% \end{enumerate}
