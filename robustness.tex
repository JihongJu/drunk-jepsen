\section{Feature transferability with label noises}
\label{sec:robustness}

%%%%%%%%%%%%%%%%%%%%%%%%%%%%%%%%%%%%%%%%%%%%%%%%%%%%%%%%%%%%%%%%%%%%%%
%%%%%%%% TEXT Formulation of transferability, synthesized noises
%%%%%%%%%%%%%%%%%%%%%%%%%%%%%%%%%%%%%%%%%%%%%%%%%%%%%%%%%%%%%%%%%%%%%%

\subsection{Problem Formulation}
\label{subsec:formulation}

\paragraph{Semantic Segmentation}

A deep learning model for semantic segmentation normally consists of two principal functions: a CNN feature extractor $g$ that extracts hierarchical feature maps $F$ from images $I$, followed by a classifier $h$ that generates pixel-by-pixel prediction to fit labels $S$.
Together they form a segmentation model $f$ to predict class probabilities for each of the pixels in a given image $I$:
$$f(I) = h(g(I))$$
Training is to find an optimal $f$ from the space of functions which minimizes a loss function $L$ that measures the distance between $S$ and $f(I)$:
$$f^{\ast} = \argmin_{f}{L(S, f(I))}$$

% $R^{h \times w \times c} \rightarrow R^{d}$
% $H: R^{d} \rightarrow R^{h \times w \times k}$
% $R^{h \times w \times c} \rightarrow R^{h \times w \times k}$,
% where $h, w$ are image height and weight respectively; $c$ is the number of image channels and $k$ is the number of classes; $d$ is the total number of extracted features.
% Prediction is then given by:
% $$P(y \vert x) = \sigma(F(x))$$
% where $\sigma$ is the sofmtax function.

\paragraph{Feature Transferability}

A pre-trained feature extractor, $g_t$, can be transferred as an initialization of the feature extractor and is expected to result in a better performed $f^{\ast}$ on the fine-tuning test set than a random choice of initialization $g_0$.
The corresponding fine-tuning performance improvement indicates the transferability of the pre-trained feature extractor.
Ideally, a feature extractor pre-trained with noisy labels would result in a fine-tuned model with equivalent performance as one pre-trained with true labels.
The difference in the result fine-tuning performance improvement between the noisy and clean pre-trained feature extractor can tell the influence of label noises on feature transferability.


\subsection{Label noise simulation}
\label{subsec:noises}

It is challenging to find a dataset with both clean and noisy labels available, so we tried to synthesize segmentation noises with correct labels.
A straightforward way to simulate noisy labels is to corrupt true labels stochastically for each segment with a corruption model.
The corruption model simply describes the probability of the observed labels given the true labels.
% $$p(\tilde{y_{ij}} \vert x, y_{ij}, e_{ij})$$
% where the binary error occurence $e_{ij}$ depends on the inputs $x$ and true labels $y$.

For segmentation problems, each pixel (or voxel for 3D segmentation) of a training image has a label assigned to one of the pre-defined categories.
Supposing there are $K$ pre-defined categories, the label of a pixel in the $i$-th row and $j$-th column from an image of size $h \times w$:
\[
  y_{ij} =
    \begin{cases}
      1 \leq k \leq K, & \text{for foreground pixels} \\
      0, & \text{for background pixels}
    \end{cases}
\]
where $1 \leq i \leq h, 1 \leq j \leq w$ and $i,j,k \in \mathrm{Z}^+$.


\textbf{Inexhaustive segmentation}

Given the true labels, which pixels belong to the same object is known.
Inexhaustive segmentation can be synthesized as flipping all pixels for an object of the $k$-th category from $k$ to $0$ with probability $p_{0k}$.
The probability for these pixels to have correct labels is then: $1-p_{0k}$.

% $$p(\tilde{y_{ij}}=0\vert y_{ij}=k) = p_{0k}$$
% $$p(\tilde{y_{ij}}=k\vert y_{ij}=k) = 1-p_{0k}$$

\textbf{Misclassification}

To synthesize misclassification, we can stochastically convert pixel labels of segment for an object of the $k$-th category from $k$ to $j$ with probability $p_{jk}$, where $j, k \leq K$ and $j,k \in \mathrm{Z}^+$.
Probabilities for all possible combinations of $j$ and $k$ form a \textit{confusion matrix} in which probabilities in each column sum up to 1.
% $$p(\tilde{y_{ij}}=j \vert y_{ij}=k) = p_{jk}, ij \in S, k,j \in [1,K]$$
% where $\sum_{j=1}^{K}p_{jk}=1$.

\textbf{False segmentations}

The presence of false segmentations can be synthesized in two steps: excluding classes except one from the foreground categories, forming the clean labels set, and assigning pixels of the excluded categories with foreground labels with a probability of $p_{11}$, constructing the noisy labels set.% $$p(\tilde{y_{ij}}=k \vert x, y_{ij}=0), ij \in S $$
