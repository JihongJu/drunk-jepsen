\section{Problem Formulation}
\label{sec:formulation}
%%%%%%%%%%%%%%%%%%%%%%%%%%%%%%%%%%%%%%%%%%%%%%%%%%%%%%%%%%%%%%%%%%%%%%
%%%%%%%% TEXT Formulation of segmentation, PU learning
%%%%%%%%%%%%%%%%%%%%%%%%%%%%%%%%%%%%%%%%%%%%%%%%%%%%%%%%%%%%%%%%%%%%%%
In this section, we formulate the model for semantic segmentation and learning with positive and unlabeled data (PU Learning).

\paragraph{Model for semantic segmentation}
A deep learning model for semantic segmentation normally consists of two principal functions: a CNN feature extractor $g$ that extracts hierarchical feature maps $F$ from images $I$, followed by a classifier $h$ that generates pixel-by-pixel prediction to fit labels $S$.
Together they form a segmentation model $f$ to predict class probabilities for each of the pixels in a given image $I$:
\begin{equation}
  f(I) = h(g(I)).
\end{equation}
Training is to find an optimal $f$ from the space of functions which minimizes a loss function $L$ that measures the distance between $S$ and $f(I)$:
\begin{equation}
  f^{\ast} = \argmin_{f}{L(S, f(I))}.
\end{equation}
The corresponding optimal feature extractor, a.k.a., representations, $g^{\ast}$ from $f^{\ast}(I)=h^{\ast}(g^{\ast}(I))$ can be used as the initialization of $g$ for another dataset.

% $R^{h \times w \times c} \rightarrow R^{d}$
% $H: R^{d} \rightarrow R^{h \times w \times k}$
% $R^{h \times w \times c} \rightarrow R^{h \times w \times k}$,
% where $h, w$ are image height and weight respectively; $c$ is the number of image channels and $k$ is the number of classes; $d$ is the total number of extracted features.
% Prediction is then given by:
% $$P(y \vert x) = \sigma(F(x))$$
% where $\sigma$ is the sofmtax function.

% \paragraph{Transferring representations}
% The initialization of feature extractor, $g_0$, is transferr when optimizing $f$
% The corresponding fine-tuning performance improvement indicates the transferability of the pre-trained feature extractor.
% Ideally, a feature extractor pre-trained with noisy labels would result in a fine-tuned model with equivalent performance as one pre-trained with true labels.

%%%%%%%%%%%%%%%%%%%%%%%%%%%%%%%%%%%%%%%%%%%%%%%%%%%%%%%%%%%%%%%%%%%%%%
%%%%%%%% TEXT PU Learning setup
%%%%%%%%%%%%%%%%%%%%%%%%%%%%%%%%%%%%%%%%%%%%%%%%%%%%%%%%%%%%%%%%%%%%%%

\paragraph{Learning with only positive and unlabeled data}
Consider a dataset containing $N$ training examples $(x_i,y_i,s_i), i \in \{1, 2, \dots, N\}$, where $x_i$ is the observed features for the $i$-th examples, and $y_i \in \{-1,+1\}$ is the label for the $i$-th example, and $s_i$ is a binary variable denoting whether the label for the $i$-th example is observed or not.



In a normal binary classification setup, labels for all examples are observed:
\begin{equation}
  s_i=1, \forall i\in\{1,2,\dots,N\}
\end{equation}
However, in a PU learning setup, only a subset of the positive examples $P$ are observed while the rest are not:
\begin{eqnarray}
s_i =
  \begin{cases}
    1, & y_i = +1 \land i\in P\\
    0, & \text{otherwise}.
  \end{cases}
\end{eqnarray}
The set of labeled positive examples is called the positive $P$ set and the other examples form an unlabeled $U$ set.
The problem of training classifier with the P set and the U set only is referred to as \textbf{PU learning}.

In this work, we train classifiers by assigning examples in the unlabeled set negative labels and covert PU learning to a learning problem with reliable positive labels and unreliable negative loss.
The observed label $\tilde{y_i}$ for the $i$-th example is:
\begin{eqnarray}
\tilde{y_i} =
  \begin{cases}
    +1, & y_i = +1 \land i\in P\\
    -1, & \text{otherwise}.
  \end{cases}
\end{eqnarray}
